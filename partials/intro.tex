There has been a massive advancement in information technology in recent times. The number of people using the internet has also increased in the past decade. It is estimated that about 4.95 billion (62.5\%) people have used the internet worldwide in January 2022, increasing by 192 million in the past 12 months (equivalent to more than 500,000 new users each day) \cite{datareportal_stats}. The amount of data generated through the internet has also grown exponentially. It is estimated that, on average, about 2.5 quintillion bytes of data were created daily in 2020 and that 463 exabytes of data will be generated on average each day\eat{by people} as of 2025 \cite{techjury}.

The growth of the internet has given birth to social media platforms like Twitter, Reddit and Instagram, which allows people worldwide to share their thoughts extensively and have transformed the way information gets propagated today. It is estimated that the number of global users in social media was around 4.62 billion (58.4\%) in January 2022, increasing by 10.1\% over the past 12 months (equivalent to a rate of over 13 new users per second) \cite{datareportal_stats}. It has been reported that the average global user spends 2 hours 27 minutes on social media each day \cite{datareportal_stats}. In 2020, about half a million new tweets were posted every day, and about four petabytes of data were generated on Facebook every day \cite{techjury}.

Undoubtedly, these platforms have facilitated a convenient exchange of dialogues and information outreach. On the positive side, these platforms have paved the way for some positive societal changes through online social revolutions. On the other hand, the freedom, easy accessibility and anonymity that these platforms offer have put a big question mark on the integrity of different kinds of posts that appear. Some individuals or groups misuse the platform. Hence, identifying and stopping the dissemination of aggressive and harmful content is vital to safeguard the interests of different groups around the world.

One such significant issue to tackle is the spread of Hate Speech. According to Cambridge Dictionary\footnote{\url{https://dictionary.cambridge.org/us/dictionary/english/hate-speech
}}, ``\emph{Hate Speech}'' is defined as ``public speech that expresses hate or encourages violence toward a person or group based on something such as race, religion, sex, or sexual orientation''. Such messages usually target a particular social group basis religion, caste, colour, race, national origin, sex, sexual orientation, etc., such as women, Muslims, Jews, African-Americans and 
members of the lesbian, gay, bisexual and transgender (LGBT) community.\eat{The legal definitions of Hate Speech also vary from country to country.} A study reported that Online Hate Speech has risen by 20\% in the UK and US since the start of the pandemic \cite{baggs_2021}.

%%% \sdcomment{TODO: add some examples here...?}

% \eat{The March 2012 attack of a Jewish school in Toulouse, France demonstrates exactly how the subculture of hate that is proliferated online can have very real consequences offline. In this instance, a lone gunman entered a Jewish school and opened fire, killing one teacher and three children [http://www.bbc.co.uk/news/world-us-canada-17426313] }

% \eat{https://www.justice.gov/hatecrimes/hate-crimes-case-examples}

% \eat{lynchings in the American south, apartheid in South Africa, or the Holocaust}

% \eat{On 31 May 2016, Facebook, Google, Microsoft, and Twitter, jointly agreed to a European Union code of conduct obligating them to review ``[the] majority of valid notifications for removal of illegal hate speech'' posted on their services within 24 hours.}

Due to the constantly evolving cyberspace and infeasibility to manually verify every content uploaded, Hate Speech propagation has gained new momentum in the recent times, challenging both the policymakers and the research community. Despite several developments and measures to curb the spread of Hate Speech, the problem is far from being solved. Hence, Automatic Hate Speech Detection remains an active and critical area of research in NLP.

In this work, we focus on identifying those user accounts on Twitter that are keen to spread Hate Speech.
This thesis is organized as follows.
We discuss the dataset we use in Chapter \ref{chap:dataset} and some previous developments in Hate Speech detection in Chapter \ref{chap:related-works}. Then, we build some baselines and present our proposed model in Chapter \ref{chap:models}. Later, we analyze their performance and perform an error analysis in Chapter \ref{chap:results}. Lastly, we conclude and discuss some possible future works in Chapter \ref{chap:future-works}.




% We discuss the dataset we use in Chapter \ref{chap:dataset} and some previous work based on this dataset and other datasets for Hate Speech in Chapter \ref{chap:related-works}. Then, we develop some baselines and present our proposed model in Chapter \ref{chap:models}. Later, we analyze their performance and perform an error analysis in Chapter \ref{chap:results}. Lastly, we conclude and discuss some possible future works in Chapter \ref{chap:future-works}. %%TODO:: We plan to make all source code available at \url{https://github.com/subhalingamd/btp-hate-speech-profiling}.
% We make all source code available at \url{https://github.com/subhalingamd/btp-hate-speech-profiling} \sdaltcomment{private repo}.
